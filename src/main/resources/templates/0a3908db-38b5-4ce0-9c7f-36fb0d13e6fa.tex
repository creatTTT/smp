\documentclass[answers]{BHCexam}
\usepackage{hyperref}
\usepackage{graphicx}
\pagestyle{plain}
\linespread{1.5}
\setlength{\parindent}{0pt}
\begin{document}
\title{标题}



\date{}
\maketitle
\begin{groups}
\group{选择题}{}
\begin{questions}[p]
\question[] 某小组6名同学的身高情况如下表,平均身高是\key{A}cm。\( \displaystyle \frac{1}{1\times 2}+\frac{1}{2\times 3}+\frac{1}{3\times 4}+\cdots +\frac{1}{2018\times 2019}\)=\key{A}。 
\threechoices {\( \displaystyle \frac{1}{1\times 2}+\frac{1}{2\times 3}+\frac{1}{3\times 4}+\cdots +\frac{1}{2018\times 2019}\)=( )。}{46}{aa\quad\includegraphics{../../images/98-01-06-13.files/5d2ba1ab-8a8a-4b47-8771-2a2718c83cdfimage1.png}} 
\begin{solution}{4cm} 
\methodonly 设有\( x\)辆汽车,用第一种方案表示这批货物的吨数是(\( 4x+2\))吨,用第二种方案表示这批货物的吨数是(\( 5x-3\))吨。根据两种方案,货物的总吨数不变,列方程为\( 4x+2=5x-3\),解得\( x=5\),即有5辆汽车,那么这批货物一共有4×5+2=22(吨); 故选B。 
\end{solution} 
\question[] 如果每辆装4吨,就还有2吨货物不能被运走;如果每辆装5吨,装完这批货物后,其中一辆车还差3吨装满,这批货物有\key{B}吨。 
\fourchoices {\quad\includegraphics{../../images/98-01-06-13.files/5d2ba1ab-8a8a-4b47-8771-2a2718c83cdfimage1.png}}{1000}{\quad\includegraphics{../../images/98-01-06-13.files/5d2ba1ab-8a8a-4b47-8771-2a2718c83cdfimage1.png}}{9090} 
\begin{solution}{4cm} 
\methodonly 设有\( x\)辆汽车,用第一种方案表示这批货物的吨数是(\( 4x+2\))吨,用第二种方案表示这批货物的吨数是(\( 5x-3\))吨。根据两种方案,货物的总吨数不变,列方程为\( 4x+2=5x-3\),解得\( x=5\),即有5辆汽车,那么这批货物一共有4×5+2=22(吨); 故选B。 
\end{solution} 
 
\end{questions}

\end{groups}
\end{document}


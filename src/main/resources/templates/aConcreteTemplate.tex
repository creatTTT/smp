\documentclass[answers]{BHCexam}
\usepackage{hyperref}
\usepackage{graphicx}
\linespread{1.5}
\setlength{\parindent}{0pt}
\begin{document}
	
% 第一行主标题
\title{BHCexam试卷排版宏包}
% 第二行主标题
\subtitle{样例}
% 考试说明
\notice{满分100分, 10分钟完成.}
% 命题人信息
\author{微信关注公众号:橘子数学}
% 考试日期
\date{2019.12.1}
% 生成试卷头
\maketitle
	
\begin{groups}
		
% 第一个题组,显示分值,不预留空间
\group{填空}{本题组共1小题,共30.0分}
\begin{questions}[]
			
% 填空题,两个空
\question[30] 橘子数学的网址是\key{www.mathcrowd.cn}, \quad\includegraphics{screenshot002.png}\quad\
\includegraphics{screenshot002.png}橘子数学的微信公众号是\key{mathcrowd}.
\begin{solution}{4cm}
	\methodonly qwqwr柔荑花$\ln{x}$.
\end{solution}
\end{questions}
		
% 第二个题组,显示分值,不预留空间
\group{选择}{本题组共2小题,共40.0分}
\begin{questions}[p]
			
% 选择题,四个选项
\question[30] 以下哪一项不是橘子数学社区的宗旨\key{C}.
\fourchoices{开放}{高效}{无视版权}{合作}
			
% 解答,4cm 参数被忽略
\begin{solution}{4cm}
	\method 橘子数学社区的宗旨是开放、高效、合作、变革.
	\method 见 \url{http://docs.mathcrowd.cn/zh_CN/latest/community/principles.html}
	\method 题干:一根绳子长\( \dfrac{6}{7}\)米,用去\( \dfrac{1}{5}\)米后,还剩(   )米。
	选项:A. \( \dfrac{23}{35}\)            B. \( \dfrac{1}{2}\)            C. \( \dfrac{24}{35}\)
	答案:A
	解析:求还剩多少米,用减法计算,\( \dfrac{6}{7}-\dfrac{1}{5}=\dfrac{30}{35}-\dfrac{7}{35}=\dfrac{23}{35}\)(米)。\\
\end{solution}
			
% 选择题,五个选项
\question[40] 以下数学公式显示有明显瑕疵的是\key{D}.
	\fivechoices{$\sin A$}{$2+3\mathrm{i}$}{$x^2$}{$\ln x$}{$\mathrm{e}^{\mathrm{i}\theta}$}
			
\begin{solution}{4cm}
\methodonly D 中正确的公式显示效果为$\ln{x}$.
\end{solution}
\end{questions}
		
% 第三个题组,显示分值,预留空间
\group{主观题}{}
\begin{questions}[st]
% 简答题,两个小问
	\question[30] 请回答以下问题:
	\begin{subquestions}
	\subquestion 你觉得有必要创建这样一个试题社区吗? 为什么?
	\subquestion 你对社区的建设有什么建议.
	\end{subquestions}
				
	% 解答,学生版会预留8cm的答题空间.
	\begin{solution}{8cm}
	\methodonly 欢迎加入用户群组发言讨论.
	telegram 交流群组: https://t.me/mathcrowd	
	QQ 群: 319701002
	Github项目页: \url{https://github.com/mathedu4all/mathcrowd-community/wiki}
	解析:设用了\( \displaystyle x\)周后,白色粉笔已经用完,还剩20盒彩色粉笔。那么原来白色粉笔有\( 30x\)盒,彩色粉笔有(\( 8x+20\))盒。根据原来白色粉笔是彩色粉笔的3倍,列方程为\( 30x=3\times (8x+20)\),解得\( x=10\),即用了10周,白色粉笔用完了。那么学校购买了白色粉笔10×30=300(盒); 故选D。\\
	
									
	\end{solution}

\end{questions}
		
\end{groups}
\end{document}
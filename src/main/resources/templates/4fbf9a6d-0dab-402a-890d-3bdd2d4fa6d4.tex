\documentclass{BHCexam}
\usepackage{hyperref}
\usepackage{graphicx}
\pagestyle{plain}
\linespread{1.5}
\setlength{\parindent}{0pt}
\begin{document}
\title{试卷标题}
\subtitle{一个副标题}
\notice{opooooo}
\author{123}
\date{2020-2-2}
\maketitle
\begin{groups}
\group{选择题}{}
\begin{questions}[p]
\question[] 下面的计算正确吗?\key{A} \quad\includegraphics{images/98-01-06-13.files/5d2ba1ab-8a8a-4b47-8771-2a2718c83cdfimage1.png}.某小组6名同学的身高情况如下表,平均身高是\key{A}cm。\( \displaystyle \frac{1}{1\times 2}+\frac{1}{2\times 3}+\frac{1}{3\times 4}+\cdots +\frac{1}{2018\times 2019}\)=\key{A}。 
\threechoices {789789}{46}{456456} 
\begin{solution}{4cm} 
\methodonly 设有\( x\)辆汽车,用第一种方案表示这批货物的吨数是(\( 4x+2\))吨,用第二种方案表示这批货物的吨数是(\( 5x-3\))吨。根据两种方案,货物的总吨数不变,列方程为\( 4x+2=5x-3\),解得\( x=5\),即有5辆汽车,那么这批货物一共有4×5+2=22(吨); 故选B。 
\end{solution} 
\question[] 某小组6名同学的身高情况如下表,平均身高是\key{A}cm。\( \displaystyle \frac{1}{1\times 2}+\frac{1}{2\times 3}+\frac{1}{3\times 4}+\cdots +\frac{1}{2018\times 2019}\)=\key{A}。 
\threechoices {\( \frac{3090}{100}\)\( \frac{3090}{100}\)}{\( \frac{3090}{100}\)}{\( \frac{3090}{100}\)} 
\begin{solution}{4cm} 
\methodonly 设有\( x\)辆汽车,用第一种方案表示这批货物的吨数是(\( 4x+2\))吨,用第二种方案表示这批货物的吨数是(\( 5x-3\))吨。根据两种方案,货物的总吨数不变,列方程为\( 4x+2=5x-3\),解得\( x=5\),即有5辆汽车,那么这批货物一共有4×5+2=22(吨); 故选B。 
\end{solution} 
\question[] 某小组6名同学的身高情况如下表,平均身高是\key{A}cm。\( \displaystyle \frac{1}{1\times 2}+\frac{1}{2\times 3}+\frac{1}{3\times 4}+\cdots +\frac{1}{2018\times 2019}\)=\key{A}。 
\threechoices {然后IG利润华农\( \frac{3090}{100}\)\( \frac{3090}{100}\)}{IP唐人街哦朋友镜头膜骗人\( \frac{3090}{100}\)}{弄丢了退款回娘家\( \frac{3090}{100}\)} 
\begin{solution}{4cm} 
\methodonly 设有\( x\)辆汽车,用第一种方案表示这批货物的吨数是(\( 4x+2\))吨,用第二种方案表示这批货物的吨数是(\( 5x-3\))吨。根据两种方案,货物的总吨数不变,列方程为\( 4x+2=5x-3\),解得\( x=5\),即有5辆汽车,那么这批货物一共有4×5+2=22(吨); 故选B。 
\end{solution} 
 
\end{questions}

\group{填空题}{}
\begin{questions}[]
\question[] 橘子数学的网址是\key{iioioi}, 一根绳子长\( \dfrac{6}{7}\)米,橘子数学的微信公众号是\key{\( \dfrac{6}{7}\)}. 
\begin{solution}{4cm}
\methodonly qwqwr柔荑花$\ln{x}$.
\end{solution}
\question[] :计算\key{\( \dfrac{1}{8}\)}:\( \dfrac{1}{8}\times \dfrac{1}{8}\div \dfrac{1}{8}\times \dfrac{1}{8}\)oiehnoig\key{\( \dfrac{1}{64}\)} 
\begin{solution}{4cm}
\methodonly 在没有括号的算式里,如果只含有加、减法或乘、除法,要按照从
\end{solution}
 
\end{questions}

\group{解答题}{}
\begin{questions}[t]
\question[] faikfi可能佛罗非能佛罗非农能佛罗非农农IBM人非能佛罗非农能佛罗非农能佛罗非农能佛罗非农能佛罗非农能佛罗非农 
\begin{subquestions}
\subquestion 1234564646 
 \subquestion vgrbft789789 
\subquestion 00.0e155vgdrfbrfe[[[[[[ 
 
\end{subquestions}
\begin{solution}{8cm}
\methodonly nfbv nodlfn弄弄哦if年河北 
\end{solution}
\question[] 2222222222222faikfi可能佛罗非能佛罗非农能佛罗非农农IBM人非能佛罗非农能佛罗非农能佛罗非农能佛罗非农能佛罗非农能佛罗非农 
\begin{subquestions}
\subquestion 1234564646 
 \subquestion vgrbft789789 
\subquestion 00.0e155vgdrfbrfe[[[[[[ 
 
\end{subquestions}
\begin{solution}{8cm}
\methodonly 2222222222nfbv nodlfn弄弄哦if年河北 
\end{solution}
 
\end{questions}

\end{groups}
\end{document}


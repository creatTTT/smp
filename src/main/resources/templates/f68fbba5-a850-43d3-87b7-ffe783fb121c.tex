\documentclass [UTF8]{ctexart}
\usepackage{graphicx}
\pagestyle{plain}
\linespread{1.5}
\setlength{\parindent}{0pt}
\begin {document}
\title{标题}
\date{2020.2.2}
\maketitle
1. 下面的计算正确吗?(  ) \quad\includegraphics{../../images/98-01-05-10.files/df1435fc-bb9f-429b-928a-5bdfdb5dcd1cimage2.png}.某小组6名同学的身高情况如下表,平均身高是( )cm。\( \displaystyle \frac{1}{1\times 2}+\frac{1}{2\times 3}+\frac{1}{3\times 4}+\cdots +\frac{1}{2018\times 2019}\)=( )。\\
A. 789789   B. 46   C. 456456   \\
答案:A\\
解析:设有\( x\)辆汽车,用第一种方案表示这批货物的吨数是(\( 4x+2\))吨,用第二种方案表示这批货物的吨数是(\( 5x-3\))吨。根据两种方案,货物的总吨数不变,列方程为\( 4x+2=5x-3\),解得\( x=5\),即有5辆汽车,那么这批货物一共有4×5+2=22(吨); 故选B。\\
\\
2. 某小组6名同学的身高情况如下表,平均身高是( )cm。\( \displaystyle \frac{1}{1\times 2}+\frac{1}{2\times 3}+\frac{1}{3\times 4}+\cdots +\frac{1}{2018\times 2019}\)=( )。\\
A. \( \frac{3090}{100}\)\( \frac{3090}{100}\)   B. \( \frac{3090}{100}\)   C. \( \frac{3090}{100}\)   \\
答案:A\\
解析:设有\( x\)辆汽车,用第一种方案表示这批货物的吨数是(\( 4x+2\))吨,用第二种方案表示这批货物的吨数是(\( 5x-3\))吨。根据两种方案,货物的总吨数不变,列方程为\( 4x+2=5x-3\),解得\( x=5\),即有5辆汽车,那么这批货物一共有4×5+2=22(吨); 故选B。\\
\\
3. 某小组6名同学的身高情况如下表,平均身高是( )cm。\( \displaystyle \frac{1}{1\times 2}+\frac{1}{2\times 3}+\frac{1}{3\times 4}+\cdots +\frac{1}{2018\times 2019}\)=( )。\\
A. 然后IG利润华农\( \frac{3090}{100}\)\( \frac{3090}{100}\)   \\
B. IP唐人街哦朋友镜头膜骗人\( \frac{3090}{100}\)   \\
C. 弄丢了退款回娘家\( \frac{3090}{100}\)   \\
答案:A\\
解析:设有\( x\)辆汽车,用第一种方案表示这批货物的吨数是(\( 4x+2\))吨,用第二种方案表示这批货物的吨数是(\( 5x-3\))吨。根据两种方案,货物的总吨数不变,列方程为\( 4x+2=5x-3\),解得\( x=5\),即有5辆汽车,那么这批货物一共有4×5+2=22(吨); 故选B。\\
\\
 
\end{document}

